%!TeX program=lualatex

\documentclass[UTF8]{ctexart}
\usepackage{amsmath}
\usepackage{cite}
\usepackage{comment}

\begin{document}

\section{中微子观测站}
全世界范围内有大量的中微子实验正在进行,而相当一部分的探测器只对太阳中微子,大气中微子,和经由核反应人工
制造的中微子进行探测,并且主要的研究方向是中微子振荡。只有一部分具有探测宇宙中微子,即来自超新星,活跃星系
核等中微子源的中微子的能力。下面列举一些至少已经部分完成建造,且已有数据采集的具有宇宙中微子探测能力的探测器。

\subsection{IceCube}
位于南极的冰立方(IceCube)项目在冰下的探测器由 80 列,每列 60 个数子光学模块(DOM)埋在南极冰盖下 1450m 至
2450m 处组成。DOM 包含光电倍增管,用于探测穿过冰层的相对论性带电粒子所产生的蓝色和近紫外切伦科夫光
\cite{abbasi2009icecube}。

IceCube 主要研究能量从 100GeV 到远达 EeV,TeV 范围内的中微子,并具有很高的灵敏度。其主要科学目标是绘制高能中微子
天空图,这有望同时包含弥散的中微子流和点源。已有相关的成果显示出 IceCube 数据中探测到的中微子点源,如来自
耀变体 TXS 0506+056 的中微子事件与其伽马射线事件重合\cite{icecube2018neutrino},以及星系 NGC 1068 作为潜在
的中微子源\cite{icecube2022evidence}。

\subsection{Super-K}
位于日本神冈矿山深处的超级神冈探测器(Super-K)核心为一装有 5 万吨高纯水的巨大圆柱形不锈钢罐。探测器设计分
为内外两个区域,内部探测器包含 32,000 吨水,内壁安装了 11,146 个朝内的50厘米直径的大型光电倍增管。
外部探测器包围着内部探测器,包含18,000吨水,安装了1,885个朝外的20厘米
直径的 PMT,其主动否决掉进入的宇宙射线 $\mu$ 子等粒子,同时作为周围岩石产生的中子和伽马射线的被动屏蔽层
\cite{fukuda2003super}。

Super-K 主要研究 MeV 至 GeV 范围内的中微子,最重要的探测成果是证明了中微子振荡的存在,也即证明了中微子存在
质量,为标准模型的完善作出了巨大贡献\cite{fukuda1998evidence}。虽然从建成至今尚未观察到超新星爆发事件,但
Super-K 具备探测的能力,并且还在深测超新星爆发的遗留中微子背景上作出了尝试\cite{kneale2024supernova}。

\subsection{Kamiokande}
神冈探测器(Kamiokande)是 Super-K 的前身,其已于 1995 年退役。利用 Kamiokande 人类首次直接观测到来自太阳系外的中微子,
即来自大麦哲伦星系的超新星SN1987A爆发的中微子事件\cite{hirata1988observation}。

\subsection{KM3NeT/ARCA}
KM3NeT/ARCA 探测器位于地中海海床下3500米深处,是一由数字光学模块(DOM)组成的巨大三维阵列。
每一球形的数字光学模块都集成了31个光电倍增管,类似 IceCube,它们安装于多条垂直列之上,最终
这些列成一个总体积达到1立方千米的探测器。
其探测能力覆盖了从 TeV 起的高能范围,设计目标是发现并确认这些高能中微子的天体物理源。
由于其地理位置,KM3NeT/ARCA对南天球的观测视野极佳,这对于研究银河系平面等潜在的中微子源区域尤为重要
\cite{km3net2024astronomy}。

虽然 KM3NeT/ARCA 尚在建设之中,但已安装完成的探测器已经探测到能量可能高达 220PeV 的中微子信号
\cite{km3net2025observation}。

\subsection{Baikal-GVD}
贝加尔湖吉吨级体积探测器(Baikal-GVD)是于俄罗斯西伯利亚贝加尔湖建造中的大型水下中微子望远镜。
截至2023年初的冬季部署任务后,Baikal-GVD 已经拥有96列探测器,安装了总计3456个光学模块(同样地,排列类似 IceCube)。
2018年至2022年收集的数据经分析表明,在高能级联簇射事件中存在着宇宙中微子流,这与南极的 IceCube
观测站的发现相符\cite{aynutdinov2023large}。

\section{中微子的产生}
\subsection{超新星\cite{janka2017neutrino}}
超新星爆发是宇宙中最强大的MeV中微子源之一。当大质量恒星演化到末期,其核心塌缩形成
中子星或黑洞时,巨大的引力能被释放出来,其中约99\%的能量是通过中微子的形式带走。
新形成的中子星的引力束缚能可以通过一个简单的牛顿引力模型来近似估算:
\begin{equation}
  E_b \sim E_g \approx 3.6 \times 10^53 \left(\frac{M_{ns}}{1.5 M_\odot}\right)^2
  \left(\frac{R_{ns}}{10 \mathrm{km}}\right)^{-1} \mathrm{erg} \label{eq:nsbind}
\end{equation}
其中 $R_{ns}$ 和 $M_{ns}$ 分别是中子星的半径和质量。
在超新星爆发过程中,这部分能量主要以中微子的形式在数十秒的时间内辐射出去。
\subsubsection{产生机制}
超新星爆发与中子星形成的过程中,中微子的产生分为多个阶段,来源于不同的物理过程。

主要的过程有电子俘获过程:
\begin{align}
  e^- + (A, Z) & \to (A, Z-1) + \nu_e \\
  e^- + p      & \to n + \nu_e
\end{align}
这个过程导致了爆发初期的 $\nu_e$ 闪光。后期的过程有粒子对湮灭过程和核子轫致辐射:
\begin{align}
  e^- + e^+ & \to \nu + \bar{\nu}         \\
  N + N     & \to N + N + \nu + \bar{\nu}
\end{align}
这些过程在吸积和冷却阶段占据主导地位,是 $\nu_\mu$ 与 $\nu_\tau$ 的主要来源。
\subsubsection{中微子辐射能谱}
中微子最终的能谱形状由脱耦区域的物理条件决定。在远处观测到的中微子能谱可以被一个准热谱很好地
近似 \cite{janka2017neutrino}:
\begin{equation}
  f_\alpha(E) \propto \left(\frac{E}{\langle E \rangle}\right)^\alpha
  e^{-(\alpha + 1) E/\langle E \rangle} \label{eq:nsnuf}
\end{equation}
其中 $\langle E \rangle$ 表达平均能量,通常在10-20 MeV的范围内。而 $\alpha$ 表达谱相对于标准费米
-狄拉克谱的“收缩”程度,可以由下式算出
\begin{equation}
  \frac{\langle E^2 \rangle}{\langle E \rangle^2} = \frac{2 + \alpha}{1 + \alpha}
\end{equation}
不同味的中微子与物质的相互作用截面不同,它们在不同的半径和温度处脱耦,因此最终形成的能
谱中,$\langle E \rangle$ 与 $\alpha$ 有不同的取值。

\subsection{活跃星系核}
活跃星系核(AGN)是目前最有希望的高能(TeV-PeV)中微子源候选者。IceCube合作组已经报道
了来自西佛(Seyfert)星系 NGC 1068 方向的显著中微子超额,为AGN作为高能中微子源提供了
关键证据 \cite{icecube2022evidence}。在AGN的中心,超大质量黑洞(SMBH)的强大引力驱动
着物质的吸积过程,形成了吸积盘、冕以及相对论性喷流等极端物理环境。在这些区
域,质子等宇宙线被加速到极高能量,并通过与周围的物质或辐射场相互作用产生高能中微子\cite{stecker1991high}。
\subsubsection{$p\gamma$过程\cite{kelner2008energy}}
AGN内部,特别是在辐射场极其强烈的区域,高背景辐射的浓度很可能远超过高能核子的浓度,
使得与辐射的相互作用相比核子之间的相互作用占据主导。对于 $p\gamma$ 相互作用,产生中
微子主要通过以下过程:
\begin{equation}
  \gamma + p \to n_0 \pi^0 + n_+ \pi^+ n_- \pi^- + ... \label{eq:photonmeson}
\end{equation}
即低能辐射光子与质子相互作用后产生 $\pi$ 介子。$\pi$ 介子随后发生衰变产生下一级的产物,可能是 $\gamma$ 光子
或者轻子。主要的轻子产生过程有
\begin{equation}
  \pi^+ \to e^+ + \nu_e,\; \pi^+ \to \mu^+ + \nu_\mu,\; \mu^+ \to e^+ + \nu_e + \bar{\nu}_\mu \label{eq:pgammadecay}
\end{equation}
以及它们的反粒子共轭过程。需要注意的是 $\pi^+$ 粒子和 $\pi^-$ 产生具有不同的能量阈值,这使得在不同能段
产生的轻子能谱会有显著不同。产生轻子的能谱可以统一地表示为
\begin{equation}
  \frac{dN_l}{dE_l} = \int f_p(E_p) f_{ph}(\epsilon) \Phi_l(\eta, x) \frac{dE_p}{E_p} d\epsilon \label{eq:pgammaspect}
\end{equation}
其中
\begin{equation}
  \eta = \frac{4\epsilon E_p}{m^2_p c^4},\;\; x=\frac{E_l}{E_p}
\end{equation}
而 $f_p$ 和 $f_{ph}$ 分别是初态质子和光子的分布,$E_p$ 和 $\epsilon$ 分别是初态质子和光子的能量,相应地
$E_l$ 为我们所关子的产生的轻子的能量。
描述了单次相互作用的产物能量分布,其具体形式可以通过蒙特卡洛模拟(如 SOPHIA \cite{mucke2000monte})获得,
一个拟合的结果为 \cite{kelner2008energy}:
\begin{equation}
  \Phi_l = B_l \exp\left(-s_l\left[\ln\left(\frac{x}{x'_-}\right)\right]^{\delta_l}\right)
  \times \left[\ln\left(\frac{2}{1+y'^2}\right)\right]^\psi \label{eq:Phi_l}
\end{equation}
其中 $B_l$, $\psi$, $s_l$, $\delta_l$ 都是 $\eta$ 的函数,且除 $\psi$ 之外,其它三个参数对不同轻子有
所区别。而 $x'_\pm$ 则是运动学导出的该解适用的能量范围边界,$y' = \frac{x - x'_-}{x'_+ - x'_-}$。
各种情况下的参数都可以从 SOPHIA 模拟结果中取得,并且在\cite{kelner2008energy}中已经给出。

\subsubsection{$pp$过程\cite{kelner2006energy}}
在AGN中气体密度较高的区域,例如吸积盘、尘埃环或致密的云气,高能质子之间的非弹性碰撞同样是产生中微子的重要途径
\cite{kelner2006energy}。其过程和 $p\gamma$ 现象有所类似,高能质子的非弹性碰撞可能会产生 $\pi$ 介子:
\begin{equation}
  p + p \to n_0 \pi^0 + n_+ \pi^+ n_- \pi^- + ... \label{eq:ppmeson}
\end{equation}
产生的 $pi$ 介子的能谱可以使用一个描述性的函数来表达:
\begin{equation}
  F_\pi(x, E_p) = \frac{dN_\pi}{dx} = \frac{d}{dx} \Phi(x, E_p) \label{eq:phipidef}
\end{equation}
其中的 $\Phi(x,E_p)$ 可以通过模型拟合得到,通过 QGSJET 模型有
\begin{equation}
  \Phi_{\mathrm{QGSJET}} = - B_\pi \left(\frac{1-x^\alpha}{(1+rx^\alpha)^3}\right)^4 \label{eq:phipiest}
\end{equation}
其中 $B_\pi$,$\alpha$,$r$ 都是初态质子质量 $E_p$ 的缓变函数。
同样地经过介子衰变的路径,并利用计算机模拟可以得到不同的路径下衰变得到的中微子能谱 $F_{\nu_\mu}$,$F_{\nu_e}$
\cite{kelner2006energy}。最终的中微子能谱通过将产生的 $\pi$ 介子能谱与介子衰变到中微子的能量分布函数进行卷
积得到,其能谱的具体计算和解析近似已在文献 \cite{kelner2006energy} 中详细给出。

\section{中微子天文学热门课题}
\subsection{高能中微子起源}
自IceCube发现弥散的宇宙高能中微子流以来,确定这些中微子的具体来源一直是该领域的核心问题。当前的研究正从对弥散信号
的统计分析,转向对单个或某几类特定天体源的精确识别。虽然AGN是首要候选者,但其他潜在源,如伽马射线暴(GRBs)、星暴
星系、潮汐瓦解事件(TDEs)等,也仍在积极的研究和探索中。下一代探测器(如IceCube-Gen2, KM3NeT)的主要目标就是通过提
升统计样本和角分辨率来解决这一关键问题。

\subsection{AGN 作为候选}
随着NGC 1068(一个喷流不显著的西佛星系)和TXS 0506+056(一个具有强大喷流的耀变体)被先后确定为中微子源,研究的焦点
开始转向深入探讨不同类型的AGN内部产生中微子的具体物理模型。针对IceCube观测到的中能(10-100 TeV)中微子流强于伽马射
线背景所暗示的“隐藏源”问题,AGN的致密核心区域(如吸积盘和冕)成为了研究焦点。理论认为,这些区域对高能伽马射线是不透
明的,但中微子可以穿透出来。特别是像NGC 1068这样的西佛星系,其观测到的中微子信号远强于伽马射线信号,为“隐藏核心”模
型提供了关键证据。

\subsection{多信使天文学观测}
由于中微子天文观测的统计数据量有限,将中微子探测与电磁波(射电、光学、X射线、伽马射线)以及引力波等多种“信使”的观测
相结合,已成为识别中微子源、揭示其内部物理过程的最有效手段。
快速的警报发布和全球范围内的多波段、多信使协同跟踪观测网络,使研究机构能够捕捉到更多瞬变的中微子源,完善
对宇宙极端天体事件的理解。
\bibliographystyle{ieeetr}
\bibliography{ref}
\end{document}